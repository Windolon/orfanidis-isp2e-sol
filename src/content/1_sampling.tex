\section{Sampling and Reconstruction}

\begin{itemize}

\markeditem{1.1}
The rotating wheel is the analog signal, while observations by
the flashing strobe light is akin to a sampler.

In the first case,
we have a $6$ Hz signal sampled at a frequency $f_s = 8$ Hz. As
$f = 6$ lies outside of the Nyquist interval $[-4,4]$, we obtain
$f_a = 6 \text{ mod } 8 = -2$ Hz, implying that the apparent rotational speed
is $2$ Hz but in the opposite direction.

In the second case, $f_s = 12$ Hz, right at the Nyquist limit.
The wheel "rotates" at an apparent speed of $6$ Hz, but it appears
to be flipping up and down rapidly. The rotation direction could be both ways.

In the cases of $f_s = 16$ or $24$ Hz, $f_s$ is above the Nyquist limit.
It appears to be rotating at $6$ Hz with its direction the same
as the actual wheel.

\markeditem{1.2}
$x(t)$ has frequencies:
\begin{align*}
	f_A &= 1 \text{ Hz} \\
	f_B &= 4 \text{ Hz} \\
	f_C &= 6 \text{ Hz}
\end{align*}
At $f_s = 5$ Hz, $f_A$ is within the Nyquist interval $[-2.5,2.5]$, while
$f_B$ and $f_C$ lie outside. Thus, we have:
\begin{align*}
	f_{A,a} &= 4-5 = -1 \text{ Hz} \\
	f_{B,a} &= 6-5 = 1 \text{ Hz}
\end{align*}
Then, we are ready to write down $x_a(t)$.
\begin{align*}
	x_a(t) &= 10\sin(2\pi t) + 10\sin(-2\pi t) + 5\sin(2\pi t) \\
		   &= 5\sin(2\pi t)
\end{align*}
Then,
\begin{align*}
	x_a(nT) &= x_a\left(\frac{n}{5}\right) \\
			&= 5\sin\left(\frac{2\pi n}{5}\right) \\
	x(nT)	&= x\left(\frac{n}{5}\right) \\
			&= 10\sin\left(\frac{2\pi n}{5}\right) + 10\sin\left(\frac{8\pi n}{5}\right) + 5\sin\left(\frac{12\pi n}{5}\right) \\
			&= 10\sin\left(\frac{2\pi n}{5}\right) + 10\sin\left(\frac{-2\pi n}{5}+\frac{10\pi n}{5}\right) + 5\sin\left(\frac{2\pi n}{5}+\frac{10\pi n}{5}\right) \\
			&= 10\sin\left(\frac{2\pi n}{5}\right) - 10\sin\left(\frac{2\pi n}{5}\right) + 5\sin\left(\frac{2\pi n}{5}\right) \\
			&= 5\sin\left(\frac{2\pi n}{5}\right) \\
	\Rightarrow x(nT) &= x_a(nT)\text{.}
\end{align*}

For $f_s = 10$ Hz, $f_A$ and $f_B$ is within the Nyquist interval and we
have
\begin{equation*}
	f_{C,a} = 6 - 10 = -4 \text{ Hz}
\end{equation*}
Then we have
\begin{align*}
	x_a(t) &= 10\sin(2\pi t) + 10\sin(8\pi t) - 5\sin(8\pi t) \\
		   &= 10\sin(2\pi t) + 5\sin(8\pi t)
\end{align*}
And to check that $x(nT) = x_a(nT)$:
\begin{align*}
	x_a(nT) &= x_a\left(\frac{n}{10}\right) \\
			&= 10\sin\left(\frac{2\pi n}{10}\right) + 5\sin\left(\frac{8\pi n}{10}\right) \\
	x(nT) 	&= x\left(\frac{n}{10}\right) \\
			&= 10\sin\left(\frac{2\pi n}{10}\right) + 10\sin\left(\frac{8\pi n}{10}\right) + 5\sin\left(\frac{12\pi n}{10}\right) \\
			&= 10\sin\left(\frac{2\pi n}{10}\right) + 10\sin\left(\frac{8\pi n}{10}\right) + 5\sin\left(\frac{-8\pi n}{10} + \frac{20\pi n}{10}\right) \\
			&= 10\sin\left(\frac{2\pi n}{10}\right) + 10\sin\left(\frac{8\pi n}{10}\right) - 5\sin\left(\frac{8\pi n}{10}\right) \\
			&= 10\sin\left(\frac{2\pi n}{10}\right) + 5\sin\left(\frac{8\pi n}{10}\right) \\
	\Rightarrow x(nT) &= x_a(nT)\text{.}
\end{align*}

\end{itemize}
