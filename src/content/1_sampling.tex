\section{Sampling and Reconstruction}

\begin{itemize}

\markeditem{1.1}
The rotating wheel is the analog signal, while observations by
the flashing strobe light is akin to a sampler.

In the first case,
we have a $6$ Hz signal sampled at a frequency $f_s = 8$ Hz. As
$f = 6$ lies outside of the Nyquist interval $[-4,4]$, we obtain
$f_a = 6 \text{ mod } 8 = -2$ Hz, implying that the apparent rotational speed
is $2$ Hz but in the opposite direction.

In the second case, $f_s = 12$ Hz, right at the Nyquist limit.
The wheel "rotates" at an apparent speed of $6$ Hz, but it appears
to be flipping up and down rapidly. The rotation direction could be both ways.

In the cases of $f_s = 16$ or $24$ Hz, $f_s$ is above the Nyquist limit.
It appears to be rotating at $6$ Hz with its direction the same
as the actual wheel.





\markeditem{1.2}
$x(t)$ has frequencies:
\begin{align*}
	f_A &= 1 \text{ Hz} \\
	f_B &= 4 \text{ Hz} \\
	f_C &= 6 \text{ Hz}
\end{align*}
At $f_s = 5$ Hz, $f_A$ is within the Nyquist interval $[-2.5,2.5]$, while
$f_B$ and $f_C$ lie outside. Thus, we have:
\begin{align*}
	f_{A,a} &= 4-5 = -1 \text{ Hz} \\
	f_{B,a} &= 6-5 = 1 \text{ Hz}
\end{align*}
Then, we are ready to write down $x_a(t)$.
\begin{align*}
	x_a(t) &= 10\sin(2\pi t) + 10\sin(-2\pi t) + 5\sin(2\pi t) \\
		   &= 5\sin(2\pi t)
\end{align*}
Then,
\begin{align*}
	x_a(nT) &= x_a\left(\frac{n}{5}\right) \\
			&= 5\sin\left(\frac{2\pi n}{5}\right) \\
	x(nT)	&= x\left(\frac{n}{5}\right) \\
			&= 10\sin\left(\frac{2\pi n}{5}\right) + 10\sin\left(\frac{8\pi n}{5}\right) + 5\sin\left(\frac{12\pi n}{5}\right) \\
			&= 10\sin\left(\frac{2\pi n}{5}\right) + 10\sin\left(\frac{-2\pi n}{5}+\frac{10\pi n}{5}\right) + 5\sin\left(\frac{2\pi n}{5}+\frac{10\pi n}{5}\right) \\
			&= 10\sin\left(\frac{2\pi n}{5}\right) - 10\sin\left(\frac{2\pi n}{5}\right) + 5\sin\left(\frac{2\pi n}{5}\right) \\
			&= 5\sin\left(\frac{2\pi n}{5}\right) \\
	\Rightarrow x(nT) &= x_a(nT)\text{.}
\end{align*}

For $f_s = 10$ Hz, $f_A$ and $f_B$ is within the Nyquist interval and we
have
\begin{equation*}
	f_{C,a} = 6 - 10 = -4 \text{ Hz}
\end{equation*}
Then we have
\begin{align*}
	x_a(t) &= 10\sin(2\pi t) + 10\sin(8\pi t) - 5\sin(8\pi t) \\
		   &= 10\sin(2\pi t) + 5\sin(8\pi t)
\end{align*}
And to check that $x(nT) = x_a(nT)$:
\begin{align*}
	x_a(nT) &= x_a\left(\frac{n}{10}\right) \\
			&= 10\sin\left(\frac{2\pi n}{10}\right) + 5\sin\left(\frac{8\pi n}{10}\right) \\
	x(nT) 	&= x\left(\frac{n}{10}\right) \\
			&= 10\sin\left(\frac{2\pi n}{10}\right) + 10\sin\left(\frac{8\pi n}{10}\right) + 5\sin\left(\frac{12\pi n}{10}\right) \\
			&= 10\sin\left(\frac{2\pi n}{10}\right) + 10\sin\left(\frac{8\pi n}{10}\right) + 5\sin\left(\frac{-8\pi n}{10} + \frac{20\pi n}{10}\right) \\
			&= 10\sin\left(\frac{2\pi n}{10}\right) + 10\sin\left(\frac{8\pi n}{10}\right) - 5\sin\left(\frac{8\pi n}{10}\right) \\
			&= 10\sin\left(\frac{2\pi n}{10}\right) + 5\sin\left(\frac{8\pi n}{10}\right) \\
	\Rightarrow x(nT) &= x_a(nT)\text{.}
\end{align*}





\markeditem{1.3}
We simplify $x(t)$ using the trigonometric identity
$\sin\theta\sin\varphi = \frac{1}{2}(\cos(\theta-\varphi)-\cos(\theta+\varphi))$.
\begin{align*}
	x(t) &= \cos(5\pi t) + 4\sin(2\pi t)\sin(3\pi t) \\
		 &= \cos(5\pi t) + 2\cos(\pi t) - 2\cos(5\pi t) \\
		 &= 2\cos(\pi t) - \cos(5\pi t)
\end{align*}
Then we know $x(t)$ has frequencies:
\begin{align*}
	f_A &= 0.5 \text{ kHz} \\
	f_B &= 2.5 \text{ kHz}
\end{align*}
At $f_s = 3$ kHz, the Nyquist interval is $[-1.5,1.5]$. $f_B$ lies outside of it.
We have $f_{B,a} = -0.5$ kHz. Then,
\begin{align*}
	x_a(t) &= 2\cos(\pi t) - \cos(-\pi t) \\
		   &= \cos(\pi t)
\end{align*}
Two other aliased signals are
\begin{align*}
	x_1(t) &= 2\cos(2\pi(0.5+1\cdot 3)t) - \cos(2\pi(2.5+1\cdot 3)t) \\
		   &= 2\cos(7\pi t) - \cos(11\pi t) \\
	x_2(t) &= 2\cos(2\pi(0.5+2\cdot 3)t) - \cos(2\pi(2.5+2\cdot 3)t) \\
		   &= 2\cos(13\pi t) - \cos(17\pi t)
\end{align*}
More generally, the class of signals that are aliased with $x(t)$ under
$f_s = 3$ kHz can be obtained by replacing every frequency $f$ in $x(t)$
with $f + nf_s$, where $n$ can be any integer.
\begin{equation*}
	x_{\text{general}}(t)
	= 2\cos(2\pi(0.5+3n)t) - \cos(2\pi(2.5+3k)t)
\end{equation*}
Notice we used two different indexing variables $n$ and $k$, because
they can obviously be different and still alias to the same $x(t)$.





\markeditem{1.4}
We simplify $x(t)$ using the trigonometric identity
$\cos\theta\cos\varphi = \frac{1}{2}(\cos(\theta-\varphi)+\cos(\theta+\varphi))$.
\begin{align*}
	x(t) &= \cos(8\pi t) + 2\cos(4\pi t)\cos(6\pi t) \\
		 &= \cos(8\pi t) + \cos(2\pi t) + \cos(10\pi t)
\end{align*}
We then have $x(t)$'s list of frequencies:
\begin{align*}
	f_A &= 1 \text{ Hz} \\
	f_B &= 4 \text{ Hz} \\
	f_C &= 5 \text{ Hz}
\end{align*}
Of which, $f_B$ and $f_C$ lies outside of the Nyquist interval
$[-2.5,2.5]$. Their aliases would be
\begin{align*}
	f_{B,a} &= 4 - 5 \\
			&= -1 \text{ Hz} \\
	f_{C,a} &= 5 - 5 \\
			&= 0 \text{ Hz}
\end{align*}
Then we have the aliased signal
\begin{align*}
	x_a(t) &= \cos(-2\pi t) + \cos(2\pi t) + 1 \\
		   &= 1 + 2\cos(2\pi t)
\end{align*}

For $f_s = 9$ Hz, only $f_C$ would be aliased.
\begin{equation*}
	f_{C,a} = 5 - 9 = -4 \text{ Hz}
\end{equation*}
Then,
\begin{align*}
	x_a(t) &= \cos(8\pi t) + \cos(2\pi t) + \cos(-8\pi t) \\
		   &= \cos(2\pi t) + 2\cos(8\pi t)
\end{align*}

\end{itemize}
